\newpage
\section{Diskussion}
\label{sec:diskussion}

Die Zahl der theoretischen Trennstufen aus der Stufenkonstruktion im McCabe-Thiele-Diagramm nimmt mit sinkender Kopfproduktreinheit ab. Wo für den Versuch 1 noch 6 Stufen konstruiert wurden, waren für den Versuch 2 nur noch 5 und für den Versuch noch 4 Stufen nötig. Natürlich blieb die Bodenzahl in der Versuchskolonne gleich. Die wirkliche Anzahl der Böden liegt immer weit über der theoretischen Zahl.

Aus den Berechnungen geht hervor, dass der Kopfproduktstrom bei sinkendem Rücklaufverhältnis ansteigt. Mit dem Rücklaufverhältnis sinkt aber auch die Konzentration an Leichtsieder im Produkt.
